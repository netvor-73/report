\chapter*{Introduction}
\vspace{-10pt}
\enlargethispage{2\baselineskip}
The world nowadays is moving towards automation at an unprecedented rate. Almost every month, a new break-through is made in the path towards creating smart computer systems with the ability to learn and make decisions of their own. This progress does not only manifest itself in theoretical work and research papers, but also in real world applications such as the latest voice recognition software found in Amazons' Alexa, or the computer vision systems used by Teslas' autonomous cars. Most experts and speculators predict that soon enough, all of the aspects of ordinary life will be dependent upon the use of these artificially intelligent machines. Some are optimistic and consider this as a practical solution for a wide range of problems in various fields such as medicine, transportation, telecommunication, and even politics and economics. Others express fears that this technology may produce more problems that it would solve.
% In either case, the best way for academics to approach this is to explore this technology and develop a better understanding of the theory and methodology surrounding it.
This work is an attempt to bring a contribution to the field, and specifically the topic of deep convolutional neural networks, by developing a real world Automatic License Plate Recognition (ALPR) system to identify Algerian license plate numbers using image inputs of the cars. ALPR systems help identify vehicle license plates in an efficient manner without the need of major human resources. Recently, ALPR systems has become more and more important. It can be used by government agencies to find cars that are involved in crime, look up if annual fees are paid or identify persons who violate the traffic rules. Many countries (e.g. U.S., Japan, Germany, Italy, U.K, France, \ldots) have successfully applied ALPR in their traffic management. Several private operators are also benefiting from ALPR systems.

Recognition tasks are extremely simple for humans, whereas for computers, it rather a very tedious work since all what a computer can understand are numbers. Recent advancements in computer vision has given computers the ability to extract semantically meaningful information from images. In this project, we take advantage of this ability to develop an ALPR system specific to Algerian cars. The system is mainly split into three major stages: data collection and labeling, license plate detection and finally character (digit) detection. The two last stages uses various deep learning techniques and are further elaborated in chapter three and four along side with the first stage.
Traditional computer vision techniques employ features chosen by humans to represent the underlying features of the image. These techniques require sophisticated human-designed models to translate raw input pixels into useful recognition responses. Using a deep learning approach, these underlying human-engineered features are automatically selected by the algorithm.
The techniques used throughout the project are relatively outdated considering the high rate of innovation in deep learning. Both works showed practical state-of-the-art results in the tasks of object detection. This project also includes the creation of a labeled data set of Algerian license plates, which is the first one ever of this kind. This data set will allow for other students or researchers to conduct similar works and will be used as a benchmark to track advancements in Algerian ALPR systems.
