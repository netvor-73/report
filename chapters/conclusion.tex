\chapter*{Conclusion}
In conclusion it is reasonable to say that the Faster-RCNN and YOLO models in particular and CNNs in general are a highly effective method in addressing automation and computer vision challenges such as license plate recognition. The satisfying results obtained by both methods using relatively outdated hardware resources and methods are evidence of that. These results would also open the gate for further research and innovation on this particular application.

To recap, this work was an attempt to create a practical application to be used for Algerian license plate detection and recognition. First, the data set was built from the ground up, including both data collection and data labeling processes. Second, the Faster-RCNN and YOLO models were selected as main tools for this task. Afterwards, a series of different models with different structures and parameters were trained and analyzed in order to explore the effects of these parameters on the outcomes of the model, and to figure out the best methods in the building of the application. These models were tested on new real world data and have shown accuracies of 84.21\% and 70.3\% for Faster-RCNN and YOLO respectively. Finally, in the strive to achieve real world applications requirements, both methods have been combined.

In the future, this project can be improved by working on many aspects of development such as the ones mentioned in the last chapter. And can be refined to be a useful tool in the hands of different institutions, businesses, and even law enforcement or security organizations. As in fact, this work is being optimized by the authors and integrated into a real world application used for monitoring garbage trucks under the supervision of a tech start-up in Algiers called Brainiac. A direct quote form a Co-Founder of this company states the following : "Introducing similar technologies into the Algerian market is not only a savvy business idea, but also a great opportunity for researchers and young software developers to turn their innovations into real world applications". And finally, it would also be insightful to mention the following quote from Ray Kurzweil - American inventor and futurist - :  “Artificial intelligence will reach human levels by around 2029. Follow that out further to, say, 2045, and we will have multiplied the intelligence a billion-fold.”
