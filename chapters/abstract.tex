\chapter*{Abstract}
In some institutions, office buildings, or government facilities
the flow of incoming and outgoing traffic of people and cars needs
to be monitored and recorded for security purposes as well as practicality
and automation of entry pass for vehicles.
In this project, a data set containing 1000 car images is collected, labeled, and then split into
training set and testing set. The size of this data set would allow for a transfer learning
approach and fine tuning of models.
The next step is to train various models belonging to the Yolo and
Faster RCNN families to perform the task of plate detection only. Once the models are
trained and optimized they were used to crop images of plates from the
original images of cars. These cropped images were used to train models for the
task of digit recognition similar to those trained for plate detection. The training
process was repeated for different structures and parameters of the models in
order to obtain the best performance possible. Evaluating these models relies on
the use of the mAP (mean average precision) which is used in the original papers of
YOLO and Faster-RCNN. The evaluation of the final model (plate detection + digit recognition)
will rely on the accuracy of performing the identification of the license plate numbers.
This project aims to provide Algerian academics and software developers with
a benchmark data set for further research on the topic and for evaluation of
future models. It would also provide users of the application with a reliable and
practical security tool.
